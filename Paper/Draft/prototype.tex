\section{HashPipe Prototype in P4}
\label{sec:prototype}

%- add details of p4 prototype
%- code fragments?
%- some details of barefoot compiler; and configurations there.
%- p4 to OVS compiler?

We built a prototype of \TheSystem using P4 version 1.1~\cite{p4-v1.1-spec} and verified that our prototype produced the same results as our HashPipe simulator. We did this by running a small number of artificially generated packets on the switch behavioral model \cite{p4-bm} as well as our simulator and ensuring that the hash table was identical in both cases at the end of the measurement interval.


\begin{figure}
 \begin{lstlisting}[basicstyle=\footnotesize, caption=
  First HashPipe stage with insertion of new flow. Fields prefixed with m are metadata fields, label=q1:p4-code, captionpos=b, basicstyle=\footnotesize, breaklines = true,
numbers=left, xleftmargin=2em,frame=single,framexleftmargin=2.0em]
 action doStage1(){
   mKeyCarried = ipv4.srcAddr;
   mCountCarried = 0;
  modify_field_with_hash_based_offset(mIndex, 0, stage1Hash, 32);

   // read the key and value at that location
   mKeyTable = flowTracker[mIndex];
   mCountTable = packetCount[mIndex];
   mValid = validBit[mIndex];

   // check for empty location or different key
   mKeyTable = (mValid == 0) ? mKeyCarried : mKeyTable;
   mDif = (mValid == 0) ? 0 : mKeyTable -  mKeyCarried;

   // update hash table
   flowTracker[mIndex] = ipv4.srcAddr;
   packetCount[mIndex] = (mDif == 0) ? mCountTable+1: 1;
   validBit[mIndex] = 1;

   // update metadata carried to the next table stage
   mKeyCarried = (mDiff == 0) ? 0 : mKeyTable;
   mCountCarried = (mDiff == 0) ? 0 : mCountTable;  
}
 \end{lstlisting}
 \end{figure}

At a high level, \TheSystem uses a match-action stage in the switch pipeline for each hash table. In our algorithm, each match-action stage has a single default action---the algorithm execution---that applies to every packet. Every stage uses its own P4 \emph{register array}---stateful memory that persists across successive packets---for the hash table. In this context, every stage's P4 registers act as a hash table where the flow identifiers and the associated counts are maintained. The P4 action blocks for the first two stages are presented in Listings \ref{q1:p4-code} and \ref{q2:p4-code}; all stages from stage $2$ onward are identical to that of stage $2$. The remainder of this section walks through the P4 language constructs with specific references to their usage in our \TheSystem prototype.\\

 \noindent \textbf{Hashing to sample locations:} The action in each stage of the match-action pipeline  hashes on the flowid (\texttt{srcip} address in this case) with a custom hash function as indicated in line $4$ of Listing \ref{q1:p4-code}. This is used to pick the location where we check for the key and if absent, the flow is used as one of the samples for approximating the minimum. The P4 behavioral model \cite{p4-bm} allows customized hash function definitions. We use hash functions of the type $(a_i\cdot x + b_i)\%p$ where the chosen $a_i, b_i$ are co-prime to ensure independence of the hash functions across stages. Hash functions of this sort are also implementable on hardware and have been used in previous approaches~\cite{univmon,li2016flowradar}. 

\noindent \textbf{Registers for storing flow statistics:} The flows are tracked and updated using three registers: one to track the flow identifiers, one for the packet count, and one to test validity of a certain index. The output of the hash function is used to index on the registers for reads and writes. Register reads occur in lines $6$-$9$ and register writes occur in lines $15$-$18$ of Listing \ref{q1:p4-code}. When a particular flow id is read from the register, it is checked against the one currently being carried. Depending on whether there is a match or mismatch, the counter value written back is $1$ or the current value incremented by $1$. 

\noindent \textbf{Packet metadata for tracking current minimum:} The values read from the registers are placed in packet metadata since we cannot test conditions directly on the register values in P4. This is crucial in local minimum computation before any write-back into the register happens (lines $11$-$13$ of Listing \ref{q1:p4-code} and lines $3$, $6$, and $9$ of Listing \ref{q2:p4-code}). Apart from helping with updates in the same stage, packet metadata plays a central role in conveying state (the current minimum flow id, in this case) from one stage to another. This is then used for checks downstream. The updates that set these metadata fields across stages are similar to lines $20$-$22$ of Listing \ref{q1:p4-code}.

\begin{figure}
 \begin{lstlisting}[basicstyle=\footnotesize, caption=
Second HashPipe stage with rolling minimum. Fields prefixed with m are metadata fields, label=q2:p4-code, captionpos=b, basicstyle=\footnotesize, breaklines=true,
numbers=left, xleftmargin=2em,frame=single,framexleftmargin=2.0em]
action doStage2{
  ....
 mKeyToWrite =  (mCountInTable <  mCountCarried) ? mKeyCarried : mKeyTable));
 flowTracker[mIndex] = (mDiff == 0) ? mKeyTable : mKeyToWrite;

 mCountToWrite =  (mCountTable < mCountCarried) ? mCountCarried : mCountTable;
 packetCount[mIndex] = (mDiff == 0) ? (mCountTable + mCountCarried) :  mCountToWrite;

 mBitToWrite = (mKeyCarried == 0) ? 0 : 1);
 validBit[mIndex] = (mValid == 0) ? mBitToWrite : 1);
 .....
}
\end{lstlisting}
\end{figure}

\noindent \textbf{Conditional updates to compute local minimum:} The first stage involves conditional updates to the register to distinguish between an empty location, a match, or a mismatch for the incoming flow. However, subsequent stages must compare the flow present in the table against the flow and count in the metadata to evict ``a relative minimum'' at the end of the pipeline. To achieve this, we use a second condition to determine if an update is necessary based on the relative sizes of the two flows. %This ensure than an update happens if and only if the flow in our metadata is larger than the flow currently in the table, the flow matches the entry in the table or if the location in the table is currently empty. 
If an update is not warranted, the metadata simply continues to the next stage in the pipeline. These kinds of conditional updates are possible at line rate~\cite{domino,tpp}.

%Our implementation uses version 1.1 of P4 since we use ternary operators for our conditional updates that aren't supported in previous versions. There is no compiler, currently, to compile down P4 version 1.1 to a hardware target. However, the only addition in version 1.1 from the supported version 1.0 other than syntactic sugar around assignment statements, is the ability to perform a read, test on a condition and a consequent update to a register location as opposed to a simple read, modify, write. However, these conditional updates are possible at line-rate (\cite{domino}, \cite{tpp}) and there will be a compiler to compile them down to hardware in the near future \vls{@Srinivas,is this needed?}.

\iffalse
\begin{figure}[h]
\lstinputlisting[language=c]{stage2.c}
\caption{Second and subsequent stages in our scheme where incoming flow is compared against the flow in the location that it hashes to and a conditional update is made depending on which of the two is larger in size currently; The code is identical to the action block from Stage 1 except that the update to the hsh table (lines 18 - 22) is replaced with the conditional update in this figure}
\label{fig:Stage2P4}
\lstset{numbers=left, numberstyle=\tiny, stepnumber=1, numbersep=5pt, escapechar=`,linewidth=8cm}
\end{figure}
\fi

%Once again, the empty location condition is subsumed under the match condition since we merely need to increment the counter with the number of packets carried in either case. However, we do need a conditional update to determine whether we need to overwrite the key in the location or not in the event of a mismatch, which is why our algorithm cannot be implemented on P4 supported hardware at the moment, but will be implementable in the near future.

\iffalse
\begin{figure}[h]
\lstinputlisting[language=c]{stage1.c}
\caption{First stage in our scheme where incoming flow is placed in the location it hashes to with either value 1 (key mismatch) or current count + 1 (key match)}
\label{fig:Stage1P4}
\end{figure}
\fi

\iffalse
\begin{figure*}[h]
 \includegraphics[width=\textwidth,height=12cm]{figures/Code.pdf}
 \caption{First stage in our scheme where incoming flow is placed in the location it hashes to with either value 1 (key mismatch) or current count + 1 (key match)}
\label{fig:P4}
\end{figure*}
\fi
